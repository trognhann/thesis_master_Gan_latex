\clearpage
\phantomsection

\addcontentsline{toc}{chapter}{Tóm tắt}
\chapter*{\fontsize{13}{13}\selectfont{Tóm tắt}}
\fontsize{12}{12}\selectfont{
\noindent\textbf{Tóm tắt:}
Các vật hình cầu là những vật trang trí đẹp mắt và độc đáo. Chúng có thể được sử dụng để tạo ra những không gian sống ấn tượng và sang trọng. Các vật hình cầu có thể được làm từ nhiều chất liệu khác nhau, như thủy tinh, kim loại, gỗ, đá, sứ hay nhựa, kim loại. Chúng có thể có nhiều kích thước, màu sắc và họa tiết khác nhau, phù hợp với nhiều phong cách và sở thích khác nhau.

Ngoài ra công nghệ chế tạo hình cầu bằng kim loại là một lĩnh vực quan trọng trong ngành công nghiệp và kỹ thuật. Hình cầu bằng kim loại có nhiều ứng dụng trong các lĩnh vực như máy bay, xe hơi, động cơ, thiết bị y tế, vũ khí hay trang sức. Tuy nhiên, chế tạo hình cầu bằng kim loại không phải là một quá trình đơn giản. Nó đòi hỏi sự kết hợp giữa các phương pháp thiết kế, tính toán, gia công và kiểm tra để đảm bảo chất lượng và hiệu quả của sản phẩm.

Một trong những công nghệ tối ưu chế tạo hình cầu bằng kim loại là công nghệ sử dụng chất lỏng cao áp -Die-less Hydroforming. Công nghệ này cho phép chế tạo hình cầu bằng kim loại với độ chính xác cao và chi phí thấp. Công nghệ này dựa trên nguyên lý biến dạng của kim loại và sử dụng chất lỏng cao áp để biến đổi đều bên trong lòng hình cầu. Nội dung của Khóa luận sẽ tập trung trình bày những đặc điểm cơ bản về tính toán, tính ưu việt trong ứng dụng của Die-less Hydroforming, và các bước tiến hành chế tạo một hình cầu.



\vspace{0.5cm}
\noindent\textit{\textbf{Từ khóa:}} \textit{Quả cầu kim loại, Dập thủy lực không cối, Polyhedral Preform, Hydro-bulging, Free Hydroforming Die-less Hydroforming.}
}