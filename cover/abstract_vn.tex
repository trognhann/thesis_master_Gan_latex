\clearpage
\phantomsection

\addcontentsline{toc}{chapter}{Tóm tắt}
\chapter*{\fontsize{13}{13}\selectfont{Tóm tắt}}
\fontsize{12}{12}\selectfont{
\noindent\textbf{Tóm tắt:}
Sự bùng nổ của nội dung số và nghệ thuật kỹ thuật số, đặc biệt trong lĩnh vực Anime/Manga, kéo theo nhu cầu mạnh mẽ về các công cụ hỗ trợ sáng tạo dựa trên trí tuệ nhân tạo. Trong đó, bài toán chuyển đổi ảnh chân dung người thật sang phong cách anime 2D (photo-to-anime) vừa giàu tiềm năng ứng dụng, vừa đặt ra nhiều thách thức về chất lượng hình ảnh và khả năng giữ lại đặc điểm nhận dạng khuôn mặt. Các phương pháp truyền thống dựa trên chuyển phong cách (style transfer) hoặc các kiến trúc GAN thế hệ đầu thường gặp vấn đề nhiễu, tạo tác (artifacts) và khó đảm bảo tính ổn định khi huấn luyện.

Luận văn này đề xuất một mô hình chuyển đổi ảnh chân dung người sang phong cách anime dựa trên kiến trúc Mạng đối nghịch tạo sinh (GAN), với nền tảng là kiến trúc Double-Tail GAN (DTGAN) của AnimeGAN thế hệ mới. Mô hình sử dụng kiến trúc encoder–decoder dựa trên ResNet với generator hai nhánh (double-tail) gồm nhánh hỗ trợ tạo ảnh anime thô và nhánh chính tinh chỉnh để tạo ra ảnh anime chất lượng cao. Bên cạnh đó, luận văn áp dụng kỹ thuật chuẩn hóa Linearly Adaptive Denormalization (LADE) nhằm giảm nhiễu và làm mượt vùng màu theo phong cách anime, đồng thời xây dựng hệ thống hàm mất mát (loss) chuyên biệt, kết hợp giữa mất mát đối nghịch, mất mát nội dung/nhận dạng (dựa trên đặc trưng VGG19), mất mát phong cách và các mất mát làm mượt–chỉnh chi tiết để cân bằng giữa bảo toàn nội dung và thể hiện phong cách.

Mô hình được huấn luyện trên hai tập dữ liệu không ghép cặp gồm ảnh chân dung người thật và ảnh chân dung anime 2D, với các bước tiền xử lý như phát hiện và căn chỉnh khuôn mặt, chuẩn hóa kích thước và tăng cường dữ liệu. Kết quả thực nghiệm cho thấy mô hình đề xuất tạo ra ảnh anime có tính thẩm mỹ cao, giữ được đặc điểm khuôn mặt của đối tượng gốc và giảm đáng kể nhiễu, được xác nhận qua các chỉ số định lượng (FID, LPIPS) cũng như đánh giá định tính từ người dùng. Cuối cùng, luận văn thảo luận các hạn chế và đề xuất hướng phát triển trong tương lai như tăng mức độ điều khiển phong cách chi tiết (màu mắt, kiểu tóc, biểu cảm) và mở rộng sang bài toán chuyển đổi video–to–anime.

\vspace{0.5cm}
\noindent\textit{\textbf{Từ khóa:}} \textit{GAN, AnimeGAN, DTGAN, chuyển phong cách ảnh, ảnh chân dung, phong cách anime, LADE, VGG19.}
}