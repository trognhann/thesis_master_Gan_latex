\clearpage
\phantomsection

\addcontentsline{toc}{chapter}{{Kết luận}}
\chapter*{Kết luận}
\textbf{Kết luận}

Nghiên cứu này đã chứng minh hiệu quả vượt trội của mô hình AnimeGANv3 (DTGAN) 
thông qua phân tích lý thuyết và kết quả thực nghiệm từ các tệp log huấn luyện. 
Kiến trúc Double-Tail cho thấy khả năng phân tách rõ ràng hai nhiệm vụ “tạo hình 
thô” và “tinh chỉnh chi tiết”, giúp mô hình hội tụ nhanh chóng, thể hiện qua việc 
loss giảm mạnh từ epoch 5, đồng thời đạt chất lượng hình ảnh cao. Việc kết hợp cơ 
chế chuẩn hóa Linearly Adaptive Denormalization (LADE) với các hàm mất mát chuyên 
biệt, bao gồm Region Smoothing, đã giải quyết hiệu quả các vấn đề nhiễu hạt và nứt 
vỡ texture thường gặp ở các GAN nhẹ. Ngoài ra, với kích thước mô hình nhỏ (~1MB) và 
tốc độ suy luận nhanh, DTGAN được đánh giá là giải pháp lý tưởng cho các ứng dụng 
di động và xử lý thời gian thực.

\textbf{Hướng nghiên cứu tiếp theo}
Tập trung vào việc cải thiện chất lượng ảnh đầu ra thông qua việc tối ưu hóa các tham số.
Tinh chỉnh các tham số của hàm mất mát để giảm thiểu nhiễu và tăng cường đặc trưng phong
cách và màu sắc. Ngoài ra, cải thiện hiệu suất huấn luyện thông qua việc tối ưu hóa các 
tham số huấn luyện.

Chuyển phong cách video (Video-to-Anime Style Transfer): Mở rộng mô hình để xử 
lý video, tập trung giải quyết hiện tượng nhấp nháy giữa các khung hình 
thông qua các ràng buộc về tính nhất quán theo thời gian (temporal consistency loss).
