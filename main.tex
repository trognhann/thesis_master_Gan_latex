\documentclass[a4paper,12pt,oneside]{book}%extreport
\usepackage{times} 
% \usepackage{multibib}
\usepackage{lipsum}
\usepackage{appendix}
\usepackage[shortlabels]{enumitem}
\usepackage{booktabs}
\usepackage{rotating} % Rotating table
\usepackage{hhline}
\usepackage{colortbl}
\usepackage{makecell}
\usepackage{afterpage}
\usepackage{mathtools} %Fixes/improves amsmath
\usepackage{setspace}
\usepackage[utf8]{vietnam} 
\usepackage{amstext, amsmath,latexsym,amsbsy,amssymb, amssymb,amsthm,amsfonts,multicol, nccmath}
\usepackage[left=3cm,right=2cm,top=2.5cm,bottom=3cm,footskip=40pt]{geometry}
\usepackage{pdflscape}
\usepackage{apacite}
\usepackage{tikz}
\usepackage{array}
\newcolumntype{C}[1]{>{\centering\let\newline\\\arraybackslash\hspace{0pt}}m{#1}}
\usepackage{subfig}
\usetikzlibrary{calc}
\usepackage{tabularx}
\usepackage{multicol}
\usepackage{array,color,colortbl}
\setcounter{secnumdepth}{4}
 %\usepackage[square,numbers]{natbib}
\usepackage[square, comma, numbers, sort&compress]{natbib}
\setlength{\parindent}{1cm}
\usepackage{color}
\usepackage{indentfirst}
\usepackage{exscale,eucal}
\usepackage{fancyhdr}
\usepackage{fncychap}
\usepackage[chapter]{algorithm}
\usepackage{multirow}
\usepackage{graphicx}
\usepackage{algpseudocode}
\usepackage{tabularx,multicol,multirow,longtable}
\usepackage{etoolbox}
\usepackage{tikz}	\usetikzlibrary{calc,arrows,decorations.pathmorphing,backgrounds,positioning,fit,shapes,decorations.shapes, shapes.geometric, decorations.pathreplacing, decorations.text}
 
	\tikzstyle{block} = [draw,rectangle,thick,minimum height=2em,minimum width=2em,drop shadow,fill=blue!50]
	\tikzstyle{sum} = [draw,circle,inner sep=0mm,minimum size=2mm]
	\tikzstyle{connector} = [->,thick]
	\tikzstyle{line} = [very thick]
	\tikzstyle{branch} = [circle,inner sep=0pt,minimum size=1mm,fill=black,draw=black]
	\tikzstyle{axes} = [->,>=stealth',semithick]
	\tikzstyle{important line} = [very thick,draw=red]
	\tikzstyle{important text} = [rounded corners,fill=red!10,inner sep=1ex]
\usepackage{circuitikz}
\usepackage{pgfplots}
\pgfplotsset{compat=1.16}
\usepackage[numbers]{natbib}
\usepackage{titlesec}
\usepackage[labelsep=period]{caption}
% \usepackage{subfigure}
\usepackage{caption}
\usepackage{subcaption}

\DeclareCaptionFormat{myformat}{\fontsize{13}{15}\selectfont#1#2#3}
\captionsetup{format=myformat}
\usepackage{titletoc}
\titlelabel{\thetitle.\,\,}
\titleformat{\chapter}[display] 
  {\fontsize{14}{16}\selectfont\bfseries\centering}
  {\MakeUppercase{\chaptertitlename}\ \thechapter}{0pt}{\fontsize{14}{16}\selectfont\MakeUppercase}
\titlespacing{\chapter}{0pt}{0pt}{40pt} 
\titleformat*{\section}{\fontsize{14}{14}\selectfont\bfseries}
\titleformat*{\subsection}{\fontsize{14}{14}\selectfont\bfseries\slshape}
\titleformat*{\subsubsection}{\fontsize{14}{14}\slshape}
\titleformat*{\paragraph}{\large\bfseries}
\titleformat*{\subparagraph}{\large\bfseries}
\graphicspath {{figures/}}

\titlespacing\section{0pt}{6pt plus 4pt minus 2pt}{1pt plus 2pt minus 2pt} 
\titlespacing\subsection{0pt}{6pt plus 4pt minus 2pt}{1pt plus 2pt minus 2pt}
\titlespacing\subsubsection{0pt}{6pt plus 4pt minus 2pt}{0pt plus 2pt minus 2pt}
\titlespacing\subsubsubsection{0pt}{6pt plus 4pt minus 2pt}{0pt plus 2pt minus 2pt}

\usepackage[subfigure]{tocloft} 

\cftsetpnumwidth{20pt}
\titlecontents{chapter}
  [0pt]
  {}
  {\fontsize{13.5}{14}\selectfont\MakeUppercase{\chaptername}\ \thecontentslabel.\,\,}
  {}
  {\cftdotfill{\cftdotsep}\contentspage} 
	\renewcommand{\cftsecaftersnum}{.}%
	\renewcommand{\cftsubsecaftersnum}{.}%

\usetikzlibrary{calc}
\newtheorem{definition}{\bf Định nghĩa}[chapter]
\newtheorem{theorem}{\bf Định lý}[chapter]
\newtheorem{lemma}{\bf Bổ đề}[chapter]

\renewcommand{\cftfigfont}{Hình~}
\renewcommand{\cfttabfont}{Bảng~ }
\usepackage{float}
\floatname{algorithm}{Thuật toán}
\makeatletter
\renewcommand\@biblabel[1]{[#1]}
\renewcommand\harvardyearleft{\unskip~(}
\renewcommand\harvardyearright{\unskip )}
\makeatother
\renewcommand{\baselinestretch}{1.4}
\flushbottom
\renewcommand*{\bibfont}{\fontsize{13}{16}\selectfont}

\usepackage[hidelinks, unicode]{hyperref}

\begin{document}
\fontsize{13}{15.5}\selectfont

\pagestyle{empty}
\input{cover/cover}
\newpage

\def\baselinestretch{1.3}
\pagestyle{plain}
\pagenumbering{gobble}
\input{cover/acknowledgement.tex}
\newpage
\clearpage
\phantomsection

\addcontentsline{toc}{chapter}{Lời cam đoan}
\chapter*{Lời cam đoan}

Tôi xin cam đoan rằng tất cả các nội dung trong tài liệu này đều là kết quả của công trình nghiên cứu cá nhân của tôi dưới sự hướng dẫn của TS.Ma Thị Châu. Tôi không sao chép bất kỳ tài liệu hay công trình nghiên cứu nào của người khác mà không chỉ rõ nguồn gốc trong phần tài liệu tham khảo. Tôi hiểu rằng việc sao chép trái phép là một hành vi vi phạm đạo đức học thuật và tôi sẽ chịu trách nhiệm về những hành vi này.

Tôi cam kết rằng, bản báo cáo của tôi không vi phạm bản quyền của bất kỳ ai và
không vi phạm bất kỳ quyền sở hữu nào, cũng như bất kỳ ý tưởng, kỹ thuật, trích
dẫn hoặc bất kỳ tài liệu nào từ công trình nghiên cứu của người khác.
Tôi xin nhận đầy đủ trách nhiệm và sẵn sàng chấp nhận mọi biện pháp kỷ luật
nếu vi phạm cam kết.

\vspace{1cm}
\hspace{7cm}\textit{Hà Nội, ngày ... tháng ... năm 2023}

\hspace{9.4cm}\textbf{Học viên}
\vspace{2.5cm}


\hspace{9.3cm}\textbf{Nguyễn Trọng Nhân}

 
\newpage
\clearpage
\phantomsection

\addcontentsline{toc}{chapter}{Tóm tắt}
\chapter*{\fontsize{13}{13}\selectfont{Tóm tắt}}
\fontsize{12}{12}\selectfont{
\noindent\textbf{Tóm tắt:}
Sự bùng nổ của nội dung số và nghệ thuật kỹ thuật số, đặc biệt trong lĩnh vực Anime/Manga, kéo theo nhu cầu mạnh mẽ về các công cụ hỗ trợ sáng tạo dựa trên trí tuệ nhân tạo. Trong đó, bài toán chuyển đổi ảnh chân dung người thật sang phong cách anime 2D (photo-to-anime) vừa giàu tiềm năng ứng dụng, vừa đặt ra nhiều thách thức về chất lượng hình ảnh và khả năng giữ lại đặc điểm nhận dạng khuôn mặt. Các phương pháp truyền thống dựa trên chuyển phong cách (style transfer) hoặc các kiến trúc GAN thế hệ đầu thường gặp vấn đề nhiễu, tạo tác (artifacts) và khó đảm bảo tính ổn định khi huấn luyện.

Luận văn này đề xuất một mô hình chuyển đổi ảnh chân dung người sang phong cách anime dựa trên kiến trúc Mạng đối nghịch tạo sinh (GAN), với nền tảng là kiến trúc Double-Tail GAN (DTGAN) của AnimeGAN thế hệ mới. Mô hình sử dụng kiến trúc encoder–decoder dựa trên ResNet với generator hai nhánh (double-tail) gồm nhánh hỗ trợ tạo ảnh anime thô và nhánh chính tinh chỉnh để tạo ra ảnh anime chất lượng cao. Bên cạnh đó, luận văn áp dụng kỹ thuật chuẩn hóa Linearly Adaptive Denormalization (LADE) nhằm giảm nhiễu và làm mượt vùng màu theo phong cách anime, đồng thời xây dựng hệ thống hàm mất mát (loss) chuyên biệt, kết hợp giữa mất mát đối nghịch, mất mát nội dung/nhận dạng (dựa trên đặc trưng VGG19), mất mát phong cách và các mất mát làm mượt–chỉnh chi tiết để cân bằng giữa bảo toàn nội dung và thể hiện phong cách.

Mô hình được huấn luyện trên hai tập dữ liệu không ghép cặp gồm ảnh chân dung người thật và ảnh chân dung anime 2D, với các bước tiền xử lý như phát hiện và căn chỉnh khuôn mặt, chuẩn hóa kích thước và tăng cường dữ liệu. Kết quả thực nghiệm cho thấy mô hình đề xuất tạo ra ảnh anime có tính thẩm mỹ cao, giữ được đặc điểm khuôn mặt của đối tượng gốc và giảm đáng kể nhiễu, được xác nhận qua các chỉ số định lượng (FID, LPIPS) cũng như đánh giá định tính từ người dùng. Cuối cùng, luận văn thảo luận các hạn chế và đề xuất hướng phát triển trong tương lai như tăng mức độ điều khiển phong cách chi tiết (màu mắt, kiểu tóc, biểu cảm) và mở rộng sang bài toán chuyển đổi video–to–anime.

\vspace{0.5cm}
\noindent\textit{\textbf{Từ khóa:}} \textit{GAN, AnimeGAN, DTGAN, chuyển phong cách ảnh, ảnh chân dung, phong cách anime, LADE, VGG19.}
}
\newpage
\clearpage
\phantomsection

\addcontentsline{toc}{chapter}{Abstract}
\chapter*{\fontsize{13}{13}\selectfont{Abstract}}
\fontsize{12}{12}\selectfont{
\noindent\textbf{Abstract:}
The explosion of digital content and digital art, especially in the Anime/Manga field, has led to a strong demand for AI-powered creative tools. Among these, the task of converting real-person portraits to 2D anime style (photo-to-anime) offers significant application potential but also poses many challenges regarding image quality and the ability to retain facial identity features. Traditional methods based on style transfer or early-generation GAN architectures often suffer from noise, artifacts, and difficulty in ensuring training stability.

This thesis proposes a model for converting real-person portraits to anime style based on the Generative Adversarial Network (GAN) architecture, specifically building upon the Double-Tail GAN (DTGAN) architecture of the new generation AnimeGAN. The model utilizes a ResNet-based encoder–decoder architecture with a double-tail generator, consisting of a branch that assists in generating rough anime images and a main branch that refines them to produce high-quality anime images. Additionally, the thesis applies the Linearly Adaptive Denormalization (LADE) technique to reduce noise and smooth color regions in anime style, while also developing a specialized loss function system that combines adversarial loss, content/identity loss (based on VGG19 features), style loss, and smoothing–detail adjustment losses to balance content preservation and style expression.

The model was trained on two unpaired datasets of real-person portraits and 2D anime portraits, with preprocessing steps such as face detection and alignment, size normalization, and data augmentation. Experimental results show that the proposed model generates aesthetically pleasing anime images, preserves the facial characteristics of the original subject, and significantly reduces noise, as confirmed by quantitative metrics (FID, LPIPS) as well as qualitative evaluation from users. Finally, the thesis discusses limitations and proposes future development directions such as increasing the level of control over detailed styles (eye color, hairstyle, expression) and extending to the video-to-anime conversion problem.

\vspace{0.5cm}
\noindent\textit{\textbf{Keywords:}} \textit{GAN, AnimeGAN, DTGAN, image style transfer, portrait, anime style, LADE, VGG19.}
}
\newpage
\pagenumbering{roman}


\setlength{\parindent}{1cm}
\setlength{\parskip}{0.6ex}

\fontsize{13}{16}\selectfont
\renewcommand{\contentsname}{\vspace{-70pt}\centerline{\fontsize{14}{16}\selectfont\MakeUppercase{Mục lục}}}
\clearpage
\phantomsection
\addcontentsline{toc}{chapter}{Mục lục}
\tableofcontents
\clearpage
% \input{cover/tomtat}
\newpage
\clearpage
\phantomsection
\addcontentsline{toc}{chapter}{Danh mục hình vẽ}
\renewcommand{\listfigurename}{\vspace{-70pt}\centerline{\fontsize{14}{16}\selectfont{\MakeUppercase{Danh mục   hình vẽ}}}}
\listoffigures
\fontsize{13}{16}\selectfont
\newpage
\clearpage
\phantomsection
\addcontentsline{toc}{chapter}{Danh mục bảng biểu}
\renewcommand{\listtablename}{\vspace{-70pt}\centerline{\fontsize{14}{16}\selectfont{\MakeUppercase{Danh mục bảng biểu}}}}
\listoftables
\fontsize{13}{16}\selectfont
\newpage

% \clearpage
% \input{cover/vietat}
% \newpage

\pagenumbering{arabic}
\pagestyle{plain}

% \input{chapter/Chap0_Introduction}
% \newpage
% \input{chapter/Chap1_Background}
% \newpage
% \input{chapter/Chap2_thietke_mophong}
% \newpage
% \input{chapter/Chap3_Main}
% \newpage
% \input{chapter/Chap4_Ketluan}
% \newpage

\appendix
% \input{cover/phuluc}
\def\baselinestretch{1}
\vspace{-2cm}
\renewcommand{\bibname}{Tài liệu tham khảo}
\clearpage
\phantomsection
\addcontentsline{toc}{chapter}{{TÀI LIỆU THAM KHẢO}}
\renewcommand{\refname}{Literary works}
\begin{thebibliography}{xx}
	\section*{Tiếng Việt}
	\vspace{0.3cm}
    
\harvarditem{{Nguyễn Tác Ánh}}{2004}{NguyenTacAnh2004}
Nguyễn Tác Ánh, ``{Giáo trình công nghệ kim loại}'', {\em Đại học sư phạm kỹ thuật TP. HỒ CHÍ MINH}, 2004.

\harvarditem{{Phạm Văn Nghệ}}{2006}{PhamVanNghe2006}
Phạm Văn Nghệ, ``{Công nghệ dập thủy tĩnh}'', {\em NXB Bách Khoa HN}, 2006.

\harvarditem{{Phạm Văn Nghệ et al.}}{2020}{PhamVanNghe2020}
Phạm Văn Nghệ, Đinh Văn Duy, Lại Đăng Giang, ``{Công nghệ gia công áp lực và thiết kế khuôn dập}'', {\em NXB Giáo dục}, 2020.

\harvarditem{{Trần Văn Địch}}{2008}{TranVanDich2008}
Trần Văn Địch, ``{Công nghệ chế tạo máy}'', {\em Nhà xuất bản khoa học kỹ thuật Hà Nội}, 2008, tr. 244-253.
	\vspace{0.5cm}	
    
	\section*{Tiếng Anh}	
	
\harvarditem{{Neugebauer}}{2007}{Neugebauer2007}
Reimund Neugebauer, ``{Hydo-umformung}'', {\em Springer}, 2007.

\harvarditem{{Yuan}}{in press}{ShijianYuan}
Shijian Yuan, ``{Modern Hydroforming Technology}'', {\em Modern Hydroforming Technology}, chapter 8, tr. 247 – 263.

\harvarditem{{Zhang et al.}}{1990}{Zhang1990}
Zhang, S., Wang, Z.R., \& Wang, T., ``{The integrally hydroforming process of spherical vessels}'', {\em International Journal of Pressure Vessels and Piping}, 42(1), 1990, pp. 111-120.

\harvarditem{{Hu \& Wang}}{2004}{HuWang2004}
Hu, W., \& Wang, Z.R., ``{Deformation analyses of the integrated hydro-bulge forming of a spheroidal vessel from different preform types}'', {\em Journal of Materials Processing Technology}, 151(1-3), 2004, pp. 275-283.

\harvarditem{{Wang et al.}}{2005}{Wang2005}
Wang, Z.R., Liu, G., Yuan, S.J., Teng, B.G., \& He, Z.B., ``{Progress in shell hydroforming}'', {\em Journal of Materials Processing Technology}, 167(2-3), 2005, pp. 230-236.

\harvarditem{{Lippold \& Kotecki}}{2005}{Lippold2005}
Lippold, J. C., \& Kotecki, D. J., ``{Welding Metallurgy and Weldability of Stainless Steels}'', {\em John Wiley \& Sons}, 2005.

% --- C. Nguồn Internet (Sắp xếp theo tên tổ chức/tiêu đề) ---
\section*{Nguồn Internet}

\harvarditem{{Các phương pháp hàn}}{n.d.}{Thietbikhangan}
``{Các phương pháp hàn cơ bản trong gia công kim loại}'', {\em thietbikhangan.vn}.

\harvarditem{{Die-Less Hydroforming}}{n.d.}{Dieless}
``{Die-Less Hydroforming of Shells}'', {\em https://www.springerprofessional.de/en/die-less-hydroforming-of-shells}.
	
\end{thebibliography}

\end{document}
